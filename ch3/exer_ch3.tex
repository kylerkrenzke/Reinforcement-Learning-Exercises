\documentclass[11pt]{article}

\usepackage[margin=1in]{geometry}

\title{Chapter 3 Exercises}
\author{Kyler Krenzke}
\date{}

\begin{document}

	\maketitle
	
	\begin{enumerate}
	
		\item \textbf{Exercise 3.1: Devise three example tasks of your own that fit into the MDP framework, identifying for each its states, actions, and rewards. Make the three
		examples as different from each other as possible. The framework is abstract and flexible and can be applied in many different ways. Stretch its limits in some way in at
		least one of your examples.}
		
		\begin{enumerate}
			
			\item \textit{Resistance training routine}: The state would be defined over the hidden variables of the human musculature. The actions would be sets/exercises to try
			to increase your muscular strength, endurance, size, etc. Rewards would be given proportional to the individual's increase in exercise benchmarks (i.e. bench press 1rm
			going up by 5lb could reward +5)
			
			\item \textit{Rocket League player}: The state space for Rocket League is defined as the position and velocity vectors for all the players and the ball, as well as
			game information such as the score and time remaining. The action space is the possible control inputs that can be used. The rewards would be given for winning the
			game.
			
			\item \textit{TV programming scheduler}: The state space is a set of possible TV show episodes, advertisers/commercials, and time slots to fit it all in. The action
			space is filled with selections of certain shows or advertisements at different times. Rewards could be given proportional to the number of total view time (viewers *
			time watched) to maximize advertising revenue.
			
		\end{enumerate}
	
	\end{enumerate}

\end{document}